\chapter{Miscellaneous activity}

\section{My semester expense}
I want to prepare my semester expense falling mainly into below categories:-
\\

\textbf{Semester fee} This include my tution fee

\textbf{Mess expense} My mess fee.

\textbf{Stationary} My stationary requirement

\textbf{Transportation} Mainly include my home trip

\textbf{Entertainment} Very less, because I am student of IIT Bombay

\textbf{Miscellaneous} Other important things
\\

Below table describe my expanse:-


\begin{table}[h!]
  \begin{center}
    \begin{tabular}{|p{4cm}|p{4cm}|}
      \hline
      \multicolumn{2}{|c|}{Per semester Expanse}\\
      \hline
      Expanse due to & Amount\\
	  \hline      
	  Semester fee & 18000\\
	  Mess expense & 20000\\
	  Stationary & 7000\\
	  Transportation & 9000\\
	  Entertainment & 5000\\
	  Miscellaneous & 4000\\
	\hline    
    \end{tabular}
    \caption{My semester expanse}
  \end{center}
\end{table}

\section{My favorite recipe}
I like to learn new cooking item. I went through recipe of preparing stuff for aloo
paratha. Below is the recipe.

\tikzstyle{startstop} = [ellipse, text centered, draw=black, fill=red!20]
\tikzstyle{process} = [rectangle, rounded corners, text width=2cm,text centered, draw=black, fill=blue!20]
\tikzstyle{decision} = [diamond, text centered, text width=2cm, draw=black, fill=blue!20]
\tikzstyle{arrow} = [thick,->,>=stealth]

\begin{figure}[h!]
\begin{center}
\begin{tikzpicture}[node distance=2cm]
\node (start) [startstop] {Start};
\node (pro1) [process, below of=start] {Boil potato in pressure cooker};
\node (pro2) [process, below of=pro1, yshift=-1cm] {Mash The potato nicely};
\node (pro3) [process, left of=pro2, xshift=-1.5cm] {Add little water};
\node (dec) [decision, below of=pro2, yshift=-1.5cm] {Has the mash softend};
\node (pro4) [process, right of=dec, xshift=3cm] {Mix with the other spices};
\node (pro5) [process, below of=dec, yshift=-2cm] {Stuff is ready for preparing Aloo Paratha};
\node (end) [startstop, below of=pro5, yshift=-0.5cm] {End};
\draw [arrow] (start) -- (pro1);
\draw [arrow] (pro1) -- (pro2);
\draw [arrow] (pro2) -- (dec);
\draw [arrow] (pro3) -- (pro2);
\draw [arrow] (dec) -| node[anchor=east] {No} (pro3);
\draw [arrow] (dec) -- node[anchor=south] {Yes} (pro4);
\draw [arrow] (pro4) |- (pro5);
\draw [arrow] (pro5) -- (end);
\end{tikzpicture}
\end{center}
\caption{Stuff preparation of Aloo Paratha}
\end{figure}

\section{Some important research ideas.}
Paper\cite{arxivPreprint} describes how word embedding helps in sarcasm detection by augmenting
the word embedding-based features to the sets of features of sarcasm. Similarity score
values returned by word embeddings, are of two categories:- similarity-based (where we consider maximum/minimum similarity score of most similar/dissimilar word pair
respectively), and weighted similarity-based (where we weight the maximum/minimum
similarity scores of most similar/dissimilar word pairs with the linear distance between
the two words in the sentence).

Paper\cite{acmSurvey} is a compilation of past work in automatic sarcasm detection. The three
milestones observed in the research so far: semi-supervised pattern extraction to iden-
tify implicit sentiment, use of hashtag-based supervision, and incorporation of context
beyond target text.

